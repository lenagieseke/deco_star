%!TEX root = ../deco_star.tex


%-------------------------------------------------------------------------
\section{Conclusion}
\label{sec:conclusion}

% Towards the goal of supporting artists in their creative work with innovative and meaningful tools, first a well defined and interdisciplinary understanding of creation processes and creativity is needed.

The overall challenge addressed in this survey is to show how to support artists in their work with meaningful control mechanisms. The investigation of controllability is put into the context of two-dimensional creative pattern designs. Procedural models and the computation of designs offer novel approaches to create content and benefits over traditional manual manufacturing. However, to provide control mechanisms that are intuitive to use and allow for individual designs is an ongoing research challenge. For more complete and meaningful solutions aspects of both, data-driven and procedural techniques are needed and must be merged to a unified whole.  

The reviewed techniques could complement each other and we hope to have further opened up the direction of bringing different approaches together and to carefully analyze and emphasize an artist-centered perspective. This is the basis for developing innovative tools that further the ability of artists to create and to creatively express themselves.

% To tackle this challenge, this survey provides first a better understanding of the creation process of an artist, means for creativity, and how to relate the identified characteristics to control mechanisms for digital tools. We dissect a creation process into overall characteristics and classify specific control mechanisms by their interaction types. 

% The analysis of the state of the art shows the capabilities of the different control mechanisms and potential trade-offs between approaches. For handling the ill-defined topic of creativity, we follow the definition of creativity as intentionally producing a novel and surprising product. We discuss means for creative control and relate specific control mechanisms to creative processes. By this we further a more objective judging of the ability of a technique to support creativity and a detailed comparison of methods. However, several aspects of the analysis still leave room for interpretation. Knowledge from other disciplines, for example in regard to the perception of visual features, might be able to contribute with valuable insights. We hope that our results inspire such research towards a quantifiable analysis of creative control.

% While the focus of this state of the art is on procedural models, various relevant data-driven approaches are integrated and work that solely specializes on creative designs highlighted. Our analysis shows that current work mainly focuses on specific and separated single aspects, which can not support overall creative creation processes. For more complete and meaningful solutions aspects of both, data-driven and procedural techniques are needed and must be merged to a unified whole.

\setcounter{secnumdepth}{0} %% no numbering
\section{Image References}
\label{sec:image_references}

\footnotesize
\sloppy

\begin{enumerate}
    \item{Manuscripts and Archives Division, The New York Public Library. 1450 - 1475. Historiated initial and another coat of arms. http://digitalcollections.nypl.org/items/510d47da-e47a-a3d9-e040-e00a18064a99}
    \item{Owen Jones. 1867. \textit{Examples of Chinese ornament selected from objects in the South Kensington museum and other collections.} London: S. \& T. Gilbert. http://archive.org/details/examplesofchines00jone}
    \item{The Miriam and Ira D. Wallach Division of Art, Prints and Photographs, The New York Public Library. 1882. Valentine cards utilizing decorative design, depicting fowers, hearts, butterflies and a tree. https://digitalcollections.nypl.org/items/510d47db-bc92-a3d9-e040-e00a18064a99}
    \item{Spencer Collection, The New York Public Library. 1910. Front doubleur. http://digitalcollections.nypl.org/items/8a6be0f9-3d78-b15e-e040-e00a180602c7}
    \item {Agnieszka Murphy. 2018. Polish folk art. 123RF, https://de.123rf.com/lizenzfreie-bilder/29119380.html?\&sti=nmw3eri7lnbl7fxnhi|\&mediapopup=29119380}
    \item{Colourbox. 2011. Frame with roses, Vector. https://www.colourbox.com/vector/frame-with-roses-vector-1286656}
    \item{Colourbox. 2016. Big set of hand drawn oral elements, Vector. https://www.colourbox.com/vector/oral-elements-vector-14556631}
    \item{Colourbox. 2013. Illustration of frame in Ukrainian folk style, Vector. https://www.colourbox.com/vector/frame-vector-6826661}
    \item{Izabela Rejke. 2011. Traditional Polish Folk Design. http://rejke.deviantart.com/art/Traditional-Polish-Folk-Design-192417774}
    \item{Colourbox. 2013. Ornamental khokhloma oral postcard with seamless stripe, Vector. https://www.colourbox.com/vector/ornamental-khokhloma-oral-postcard-vector-8445572}

    \item {William Morris and John Henry Dearle. 1910. Tree of life wall hanging.  https://i. pinimg.com/originals/f2/21/06/f2210609d8ee8f8b30d1f66ed8ff01db.jpg. Kelmscott Manor House and Collection. Lechlade, UK.}
    
    \item{Free Patterns Area. Laser cut wood ornament template. 2018. https://www.freepatternsarea.com/designs/ geometric-decorative-islamic-art-ornament-vector-design/. CC-BY-4.0 Creative Commons License}
    \item {Marcel's Kid Crafts. Celtic knot pattern. 2018. http://www.marcels-kid-crafts. com/celtic-knot-patterns.html. CC-BY-4.0 Creative Commons License.}
\end{enumerate}

% 
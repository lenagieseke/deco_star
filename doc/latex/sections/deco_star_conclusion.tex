%!TEX root = ../deco_star.tex


%-------------------------------------------------------------------------
\subsection{Conclusions}

% Towards the goal of supporting artists in their creative work with innovative and meaningful tools, first a well defined and interdisciplinary understanding of creation processes and creativity is needed.

\note[inline]{This is still a bit messy.}

The overall challenge addressed in this survey is how to support artists in their work with meaningful control mechanisms. The investigation of controllability is put into the context of two-dimensional creative pattern designs. Procedural models and the computation of designs offer novel approaches to create content and benefits over traditional manual manufacturing. However, to provide control mechanisms that are intuitive to use and allow for individual designs is an ongoing research challenge.

To tackle this challenge, this survey provides first a better understanding of the creation process of an artist, means for creativity, and how to relate the identified characteristics to control mechanisms for digital tools. We dissect a creation process into overall characteristics and classify specific control mechanisms by their interaction types. 

The analysis of the state of the art shows the capabilities of the different control mechanisms and potential trade-offs between approaches. For handling the ill-defined topic of creativity, we follow the definition of creativity as intentionally producing a novel and surprising product. We discuss means for creative control and relate specific control mechanisms to creative processes. By this we further a more objective judging of the ability of a technique to support creativity and a detailed comparison of methods. However, several aspects of the analysis still leave room for interpretation. Knowledge from other disciplines, for example in regard to the perception of visual features, might be able to contribute with valuable insights. We hope that our results inspire such research towards a quantifiable analysis of creative control.

While the focus of this state of the art is on procedural models, various relevant data-driven approaches are integrated and work that solely specializes on creative designs highlighted. Our analysis shows that current work mainly focuses on specific and separated single aspects, which can not support overall creative creation processes. For more complete and meaningful solutions aspects of both, data-driven and procedural techniques are needed and must be merged to a unified whole.
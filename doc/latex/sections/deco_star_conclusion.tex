%!TEX root = ../deco_star.tex


%-------------------------------------------------------------------------
\subsection{Conclusions}

Towards the goal of supporting artists in their creative work with innovative and meaningful tools, first a well defined and interdisciplinary understanding of creation processes and creativity is needed.

We dissect a creation process into overall characteristics and classify specific control mechanisms by their interaction types. The analysis of the state of the art shows the capabilities of the different control mechanisms and potential trade-offs between approaches. For handling the ill-defined topic of creativity, we follow the definition of creativity as intentionally producing a novel and surprising product. We establish means for creative control and relate specific control mechanisms to creative processes. By this we further a more objective judging of the ability of a technique to support creativity and a detailed comparison of methods. 

However, some aspects of the analysis framework still leave room for interpretation. Knowledge from other disciplines, for example in regard to the perception of visual features, might be able to contribute with valuable insights. We hope that our results inspire such research towards a quantifiable analysis of creative control.


% In order to achieve the goal of creative control for procedural ornamentation of specific design goals, an understanding of creativity and creative means, underlying models and the identification and classification of the related state of the art are required. Specific ornamental design goals are a perception of order through the structured repetition of elements and hierarchical compositions. Ornaments adapt to the space they fill and integrate contrasts and highlighting accents for visual appeal. This work follows the definition of creativity as intentionally producing a novel and surprising product and identifies navigation, transparency, variation and stimulation as analyzable means for engaging in creativity. While the focus of this work is on procedural models, various relevant data-driven approaches are integrated and work that solely specializes on ornamental designs highlighted. For a better understanding of the creative control capabilities of the state of the art, the control paradigms of how, what, where, when and who are analyzed and broken down into specific mechanisms with exemplars, parameterization, handling, filling, guiding and placing and their respective interactions. Our analysis shows that current work mainly focuses on specific and separated single aspects, which can not support overall creative control for ornamentation. For more complete and meaningful solutions aspects of both, data-driven and procedural techniques are needed and must be merged to a unified whole.


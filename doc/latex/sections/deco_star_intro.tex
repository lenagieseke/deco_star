%!TEX root = ../deco_star.tex


\section{Introduction}

% % WHY IS THIS AN IMPORTANT, RELEVANT AND CURRENT TOPIC 
% Digital tools for creative processes with various forms of output are indispensable for most artists. Even designs that have a distinct hand-made quality to them, such as Disney's animation Paperman~\cite{disney_2012_ppm} for example, were computer-generated, employing novel software solutions that are fully controllable by artists. Furthermore, more than three decades of research in academia have produced various control mechanisms for creation. These are often said to be artist-controllable but are less often proven to be so.

Pattern generation in computer graphics provides vast amounts of precise and aesthetically pleasing digital content. However, supporting artists with meaningful digital tools for creative content generation is an ongoing research challenge. More than three decades of research in academia have produced various control mechanisms for creation. On an algorithmic level however, most solutions focus on adding singular features and control mechanisms such as an example-based control or a brush. Little attention has been paid to an overall creative workflow, which needs to balance between giving users as much control as needed without burdening them with unwanted details. Also, techniques are often said to be artist-controllable but are less often proven to be so.

% WHY IS THIS A DIFFICULT PROBLEM
Most research has been executed without direct and continuous collaboration with artists. Moreover, large-scale user studies with suitable participants are usually impractical. Also, such interdisciplinary studies and the evaluation of artist feedback still suffer at times from undefined language and vague discussions. It is an open challenge to complement user studies with a more precise and determinative terminology that is also commonly applicable and understandable by artists.

Due to these obstacles, there is too little common understanding in the research community regarding what artist needs actually are and how mechanisms are validated. It is an important challenge to investigate creation processes from a more artist-centered perspective and to consider all tasks and efforts as a whole, from initial configuration requirements to local edits. As a further challenge, control mechanisms have to be linked to an expressive design space, which is the basis for all meaningful creative creation. 

% Furthermore, research has typically focused on solving one specific aspect while accepting significant trade-offs for other steps in a creation process.
% % WHAT DID THE OTHER RELATED TO THE PROBLEM?
% For example, the automatic targeted control of a large design space results in long computation times and non-intuitive configuration requirements (e.g., for~\cite{bourque_2004_ptm, wong_1998_cgf}). Equally, flexible creation pipelines that enable both automation and manual controllability often suffer from a restricted design space (e.g.,~\cite{santoni_2016_ggp}).  

% WHAT ARE WE GOING TO DO ABOUT IT?
% The previous statement naturally leads to general questions about artist-centered full controllability in combination with distinctive output spaces and a meaningful and applicable evaluation of digital creation tools. These considerations are based on decades of research in multiple disciplines. However, there is little common ground on what is needed as theoretical basis for an evaluation of creative control mechanisms with various open questions. 

% For example, what are relevant characteristics of a creation process with digital tools? What are specific control mechanisms, ranging from global to local and from automatic to manual, with varying levels of abstraction for their handling? How do these mechanisms relate to the stages of a creation process? What are the requirements for creative creation, and how can these be customized to the context of digital creation tools? To bring further insights, this survey investigates those problems in the scope of creative two-dimensional visual pattern generation, tackling a restricted but highly challenging and representative creation process and design space. 
For 2D creative pattern generation, the underlying regularity of pattern designs is based on a repetitive and balanced distribution of elements, usually following hierarchical structures. These characteristics can be efficiently implemented by procedural approaches because they automatically fill a space based on generative rules \cite{stava_2010_ipm}. An artist should be freed from such tedious, non-inspiring and repetitive tasks. In order to execute order, computational generation techniques are not only an easement, but they also perform in a potentially more precise and less error-prone way than a human artist. Hence, procedural representations are an ideal basis for creative pattern generation. 
% Much effort has gone into investigating control mechanisms for procedural representations within specific contexts, as procedural models are notoriously difficult to control~\cite{bourque_2004_ptm,lagae_2010_pis,gilet_2010_ias,benes_2011_gpm,lasram_2012_ssf,lasram_2012_ptp}.
However, the creative demands of, for example, laying out space-specific designs and of placing highlights must also be considered. Procedural models must be augmented, and different approaches must be unified in order to enable the control and quality of manual creation as well as the efficiency and accuracy of computation \cite{gieseke_2017_ooo}. 

With that creative pattern generation is a rich, compelling and instructive topic.
%This scope represents the overall balance between giving users as much control as needed without burdening them with unwanted details and its diverse design aspects make pattern generation a rich, compelling and instructive topic.
The review does not include any 3D modeling contexts. For 3D generation techniques, the interested reader is referred to the surveys of world building~\cite{smelik_2014_aso, aliaga_2016_ipm}, terrain modeling~\cite{galin_2019_aro} and game-specific approaches~\cite{hendrikx_2013_pcg, togelius_2011_sbp}.
%  Botanical and architectural procedural modeling, for example, are popular fields of research. There have been summarizing surveys for procedural noise~\cite{lagae_2010_sap}, landscapes~\cite{smelik_2014_aso} and urban spaces~\cite{vanegas_2009_mab} as well as in the context of games~\cite{hendrikx_2013_pcg, togelius_2011_sbp}, including even a short summary of models for ornamentation~\cite{whitehead_2010_tpd}. 
However, also for these potentially creative creation tasks, the investigation of enabling artist intent and a creative work flow is in the surveys less prominent. This could be credited to creative creation still being an ill-defined domain within the computer graphics community. 

In this survey, we discuss the state of the art for 2D, potentially aesthetically pleasing, visual pattern generation and investigate artist controllability in combination with distinctive design spaces. We base this analysis of current techniques on a well-defined analysis framework and offer commonly applicable control characteristics and associated mechanisms. Specifically this survey contributes with:

% Research questions are then structured along an axis of decreasing automation, from fully automated goal-oriented control to interactive creative control to manual experimental control.

% TODO: These still need work.
\legie{The contributions need some work. They are too verbose.}
\begin{itemize}
    \item The classification of control mechanisms as theoretical grounding, ranging from global to local and from automatic to manual, with varying levels of abstraction for their handling and the relation of these mechanisms to the stages of a creation process. 
    \item A dissection of a creative process into potentially classifiable characteristics within the context of algorithms as common ground for a discussion about enabling creativity.
    \item An interlinking analysis of the state of the art based on design features, bridging between procedural and data-driven solutions with joint classification results. 
    \item An investigation of the capabilities of mechanisms in a creative process and their potential for creative creation within a coherent workflow.
\end{itemize}

We start this survey with a summary of relevant terms and our understanding of them for this work in~\Cref{terminology}. In~\Cref{sec:taxo_control_mechanism} we then lay the theoretical groundwork for our later evaluation of control mechanisms of the state of the art by classifying general control characteristics and identifying the specific control mechanisms that associate with these characteristics. We add to this preparatory work in~\Cref{sec:creativ_means} with a basis for enabling a discussion of the creative means for a technique. Before turning to the analysis of the state of the art in~\Cref{sec:analysis}, we establish~\Cref{sec:design} the areas of investigation by describing possible design features of creative pattern generation and their underlying creation models in~\Cref{sec:models}. Lastly, we discuss future trends and give an outlook to relevant topics in~\Cref{sec:outlook} and conclude the survey in~\Cref{sec:conclusion}.
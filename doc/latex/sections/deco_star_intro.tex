%!TEX root = ../deco_star.tex
\section{Introduction}
Pattern generation in computer graphics provides vast amounts of precise and aesthetically pleasing digital content. However, despite over three decades of research, supporting artists with meaningful digital tools for creative content generation is an ongoing research challenge. On an algorithmic level however, most solutions focus on adding singular features and control mechanisms such as an example-based control or a brush. Little attention has been paid to an overall creative workflow, which needs to balance between giving users as much control as needed without burdening them with unwanted details. Also, techniques are often claimed to be artist-controllable but are less often proven to be so.

% WHY IS THIS A DIFFICULT PROBLEM
Most research has been executed without direct and continuous collaboration with artists and without large-scale user studies. Algorithms and methods are being developed mostly because they can be developed, for example because of the state of the art in deep learning research, and there is too little common understanding in the research community regarding what artist needs actually are and how mechanisms are validated.

The underlying regularity of 2D pattern designs is based on a repetitive and balanced distribution of elements, usually following hierarchical structures. These characteristics can be efficiently implemented by procedural approaches because they automatically fill a space based on generative rules~\cite{stava_2010_ipm}. The main motivation of inverse procedural models is to free the artist should from tedious, non-inspiring, and repetitive tasks. Computational generation techniques are not only an easement, but they also perform in a potentially more precise and less error-prone way than a human artist. Hence, procedural representations are an ideal basis for creative pattern generation. 
% Much effort has gone into investigating control mechanisms for procedural representations within specific contexts, as procedural models are notoriously difficult to control~\cite{bourque_2004_ptm,lagae_2010_pis,gilet_2010_ias,benes_2011_gpm,lasram_2012_ssf,lasram_2012_ptp}.
However, the creative demands of, for example, laying out space-specific designs and of placing highlights must also be considered. Procedural models must be augmented, and different approaches must be unified in order to enable the control and quality of manual creation as well as the efficiency and accuracy of computation~\cite{gieseke_2017_ooo}. 


\legie{We review recent advances in 2D pattern generation and we focus on methods that include certain level of automatic generation. We discuss procedural models, data-drive generation, design-specific pattern generation.}

% This is still badly written but hopefully gets the point across?
\legie{As theoretical grounding, we classify control mechanisms and their characteristics from the perspective of an artist from global to local and from automatic to manual, with varying levels of abstraction for their handling. As a basis for a discussion of the capabilities of control mechanisms in a creative process and their potential for creative creation, we review literature on creativity and summarize aspects that can help to understand such capabilities in the context of computer graphic publications.}

\legie{For the review of the state of the art, we organize recent techniques by design areas and the visual features they enable. For the design areas we further group related work by commonly used control types where ever possible. Then, we specifically analyse for each reference the offered control for an artist. Our review is summarized in Table ~\ref{table:control_mechanisms}, where we organize recent techniques by design areas and the visual features they enable on the y-axis and list for each work the specific control mechanisms they offer on the x-axis. Based on this detailed analysis, we then conclude the review of the state of the art with an overall discussion of the creative means for the different control mechanism types.}

\bb{Lena - can you please add a paragraph here stating what will be the rows and the columns of the table in MS Word? I am still fuzzy on this. I am shaping a version that is probably wrong. But I feel the reader needs to know now what is the paper about. }
\bb{We review recent advances in 2D pattern generation and we focus on methods that include certain level of automatic generation. We discuss procedural models, data-drive generation, design-specific pattern generation. We cross-analyze these methods from the viewpoint of control: example-based, shapes and masks, vector fields, and curves and sketches.}
The review does not focus on 3D modeling contexts. For 3D generation techniques, the interested reader is referred to the surveys of world building~\cite{smelik_2014_aso, aliaga_2016_ipm}, terrain modeling~\cite{galin_2019_aro} and game-specific approaches~\cite{hendrikx_2013_pcg, togelius_2011_sbp}.
%However, also for these potentially creative creation tasks, the investigation of enabling artist intent and a creative work flow is in the surveys less prominent. This could be credited to creative creation still being an ill-defined domain within the computer graphics community. 

%In this survey, we discuss the state of the art for 2D, potentially aesthetically pleasing, visual pattern generation and investigate artist controllability in combination with distinctive design spaces. We base this analysis of current techniques on a well-defined analysis framework and offer commonly applicable control characteristics and associated mechanisms. Specifically this survey contributes with:

% Research questions are then structured along an axis of decreasing automation, from fully automated goal-oriented control to interactive creative control to manual experimental control.


We start this survey with a summary of relevant terms~\Cref{terminology}. \Cref{sec:taxo_control_mechanism} lays the theoretical groundwork for our later evaluation of control mechanisms of the state of the art by classifying general control characteristics and identifying the specific control mechanisms that associate with these characteristics. We add to this preparatory work in~\Cref{sec:creativ_means} with a basis for enabling a discussion of the creative means for a technique. Before turning to the analysis of the state of the art in~\Cref{sec:analysis}, we establish in~\Cref{sec:design} the areas of investigation by describing possible design features of creative pattern generation and their underlying creation models in~\Cref{sec:models}. Lastly, we discuss future trends and give an outlook to relevant topics in~\Cref{sec:outlook} and conclude the survey in~\Cref{sec:conclusion}.
%!TEX root = ../deco_star.tex
\section{Introduction}
Pattern generation in computer graphics provides vast amounts of precise and quickly generated digital content. However, despite over three decades of research, supporting artists with meaningful digital tools for creative content generation is an ongoing research challenge. On an algorithmic level, most solutions focus on adding singular features and control mechanisms, such as an example-based control or a brush. Little attention, however, is paid to an overall creative workflow, which needs to strike a balance, giving users as much power as needed without burdening them with unwanted details. Often, techniques are claimed to be artist-controllable but are less often proven to be so.

% WHY IS THIS A DIFFICULT PROBLEM
This may be a consequence of research often being executed without direct and continuous collaboration with artists and without the support of large-scale user studies. Algorithms and methods are being developed mostly to add to the state of the art, for example, for building on recent progress in deep learning research, and there is little common understanding in the research community regarding what artist needs are and how mechanisms are validated.

Recent surveys have covered 3D generation in much detail, focusing on  world building~\cite{smelik_2014_aso, aliaga_2016_ipm}, terrain modeling~\cite{galin_2019_aro} and game-specific approaches~\cite{hendrikx_2013_pcg, togelius_2011_sbp}. In this report, we review recent advances in 2D pattern generation. 

The underlying regularity of 2D pattern designs is based on a repetitive and balanced distribution of elements, usually following hierarchical structures. These characteristics can be efficiently implemented by procedural approaches that arrange elements in space according to generative rules~\cite{stava_2010_ipm}. The primary motivation of, for example, inverse procedural models is to free the artist from tedious, non-inspiring, and repetitive tasks. Computational generation techniques are not only an easement, but they also perform in a potentially more precise and less error-prone way than a human artist. Hence, procedural representations are an ideal basis for creative pattern generation. However, the creative demands of, for example, laying out space-specific designs and placing visual accents must also be considered. Procedural models must be augmented, and different approaches must be unified to enable the control and quality of manual creation and the efficiency and accuracy of computation~\cite{gieseke_2017_ooo}. 

In the following, we review recent advances in 2D pattern generation and discuss procedural models, data-driven generation, and design-specific pattern generation. As theoretical grounding, we classify control mechanisms and their characteristics from the perspective of an artist from global to local and from automatic to manual, with varying levels of abstraction for their handling. As a basis for a discussion of the capabilities of control mechanisms in a creative process and their potential for innovative creation, we review the literature on creativity and summarize aspects that can help to understand such capabilities in the context of computer graphic publications.

We organize contemporary techniques by design areas and the visual features they enable. We further group related work by commonly used control types. Then, we specifically analyze for each reference the offered control for an artist. 
%BB this is too detailed for an intro. I would not put the reference to the table here. 
%Our review is summarized in Table~\ref{table:control_mechanisms}, where we organize recent techniques by design areas and the visual features they enable on the y-axis and list for each work the specific control mechanisms they offer on the x-axis. 
We conclude the review with a discussion of the creative means for the different control mechanism types. With this survey, we hope to not only categorize and summarize the state of the art meaningfully, but also to contribute to a shared vocabulary and a foundation for making it easier in the future to incorporate artists and creative tasks into algorithmic content creation pipelines.

%\bb{We review recent advances in 2D pattern generation, and we focus on methods that include a certain level of automatic generation. We discuss procedural models, data-driven generation, design-specific pattern generation. We cross-analyze these methods from the viewpoint of control: example-based, shapes and masks, vector fields, and curves, and sketches.}

%BB some people hate the overview of the paper at the end of the Introduction. I would save some space by not including it. By all means, we already described it above. 
%\legie: I actually do hate them but thought that it might help this paper as the flow of the paper is a bit unusual. But we can decide later on whether to use it or now.

%We start this survey with a summary of relevant terms~\Cref{terminology}. \Cref{sec:taxo_control_mechanism} lays the theoretical groundwork for our later evaluation of control mechanisms of state of the art by classifying general control characteristics and identifying the specific control mechanisms that associate with these characteristics. We add to this preparatory work in~\Cref{sec:creativ_means} with a basis for enabling a discussion of the creative means for a technique. Before turning to the analysis of the state of the art in~\Cref{sec:analysis}, we establish in~\Cref{sec:design} the areas of investigation by describing possible design features of creative pattern generation and their underlying creation models in~\Cref{sec:models}. Lastly, we discuss future trends and give an outlook to relevant topics in~\Cref{sec:outlook} and conclude the survey in~\Cref{sec:conclusion}.
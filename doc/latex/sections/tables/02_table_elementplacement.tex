%!TEX root = ../../../thesis.tex


\begin{table}
\tiny
% \left



\begin{tabu}{ |[1pt middleGrey] l g g l l l g g l l l g g g l l g g|[1pt middleGrey]}

    \taburulecolor{middleGrey}\hline
    \tableHeaderStyle

    \textit{CONTROLS}& 
    \multicolumn{2}{g}{INIT.} & 
    \multicolumn{3}{g}{EXEMP.} &
    \multicolumn{2}{g}{PARAM.} & 
    \multicolumn{3}{g}{HAND.} & 
    \multicolumn{3}{g}{FILL.} & 
    \multicolumn{2}{g}{GUIDE.} & 
    \multicolumn{2}{g|[1pt middleGrey]}{PLACE.}
    \\

    & 
    \sidy{Configuration} & \sidy{Initialization} & 
    \sidy{Image} & \sidy{Arrangement} & \sidy{Element} &
    \sidy{Visual Output } & \sidy{System} &
    \sidy{Visualization} & \sidy{Image} & \sidy{Sketch} &
    \sidy{Shapes} & \sidy{Masking} & \sidy{Curve} &
    \sidy{Painting} & \sidy{Directions} &
    \sidy{Element} & \sidy{Drag \& Drop}
    \\

    % Reference
    \citet{yeh_2009_dsa} & 
    % Config & Task Init & 
     & & 
    % Image
    % Element arrangement
    % Element
     & $\times$ &  &
    % Visual output
    % System/Generation
     &  & 
    % Custom visual UI
    % Image-based
    % Sketch-based
     &  &  &
    % Shapes
    % Masking
    % Curve/Strokes to fill
     &  &  &
    % Painting/Strokes to follow
    % Directions
    $\times$ &  & 
    % Element Placement
    % Element Drag&Drop
    $\times$ & $\times$
    \\

        % Reference
    \citet{guerrero_2016_pep} & 
    % Config & Task Init & 
    $\times$ &  & 
    % Image & Element arrangement & Element &
     & $\times$ &  &
    % Visual output  & System/Generation &
     &  & 
    % Custom visual UI & Image-based  & Sketch-based &
    $\times$ &  &  &
    % Shapes & Masking & Curve/Strokes to fill &
     &  &  &
    % Painting/Strokes to follow & Directions  &
     &  & 
    % Element Placement & Element Drag&Drop
    $\times$ & $\times$
    \\
    \taburulecolor{middleGrey}\hline



\end{tabu}

\bigskip

\begin{tabu}{ |[1pt middleGrey] l g g g l l g g |[1pt middleGrey]}

    \taburulecolor{middleGrey}\hline
    \tableHeaderStyle

    \textit{CREATIVE MEANS}& 
    \multicolumn{3}{g}{NAVI.} & 
    \multicolumn{2}{g}{TRANS.} &
    \multicolumn{2}{g|[1pt middleGrey]}{VARI.}   
    \\

    & 
    \sidy{Interactive} & \sidy{\newl{Control}{Quantity}} & \sidy{\newl{Navigation}{History}} & 
    \sidy{\newl{Control}{Domain}} & \sidy{\newl{Control}{Communication}} &
    \sidy{\newl{Design Space}{Size}} & \sidy{\newl{Design Space}{Openness}} 
    \\

    % Reference
    \citet{yeh_2009_dsa} & 
    % Interactive 
    % Quantity of Controls
    % Navigation History 
    $\bullet$ & $\circ$ &  &
    % Quality of controls
    % Control Communication
    $\circ$ & $\circ$ & 
    % Size design space
    % Openness of design space
    $\circ$ & $\circ$ 
    \\

    % Reference
    \citet{guerrero_2016_pep} & 
    % Interactive & Quantity of Controls & Navigation History &
    $\bullet$ & $\circ$ &  &
    % Quality of controls & Control Communication &
    $\circ$ & $\circ$ & 
    % % Size design space & Openness of design space &
    $\circ$ &  
    \\
    \taburulecolor{middleGrey}\hline
\end{tabu}
\hfill\




\caption[Interaction and creative means for placement techniques]{Interaction and creative means for techniques that allow for the placement and modification of single elements in combination with applying procedural design functionalities.}
\label{table:taxo_elementplacement}
\end{table}
    


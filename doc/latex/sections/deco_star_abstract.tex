%!TEX root = ../deco_star.tex


\begin{abstract}

% Alternative title: A Survey of Creative Pattern Generation \newline
    
% TODO: Contribution in one or (max) two sentences:
% This STAR contributes with an interlinking analysis and classification of control mechanisms on an algorithmic level within the scope of creative pattern generation and discusses their potential for a creative workflow.

% \bb{...is an ongoing research challenge and spans over various disciplines \bb{including...?}. 
%LG: such as computer science, psychology and design. %LG: or rather disciplines within computer graphics?
%LG: But I rather take it out here
Pattern generation in computer graphics provides vast amounts of precise and aesthetically pleasing digital content. However, supporting artists with meaningful digital tools for creative creation is an ongoing research challenge. On an algorithmic level, most solutions focus on adding 
% \bb{I do not understand. Can you add an example? $\Rightarrow$} 
singular features and control mechanisms such as an example-based control or a brush. Little attention has been paid to an overall creative workflow, which needs to delicately balance between giving users as much control as needed without burdening them with unwanted details. 

We examine current methods in the scope of 2D creative pattern generation and their control mechanisms. This is an intricate design problem because, on the one hand, the repetitive nature of a pattern is well-matched with algorithmic creation, especially with procedural models. On the other hand, control mechanisms beyond automation need to be given to artists for inventive designs. 
% \bb{The next paragraph could use more details. It is clear to an expert, but may be obfuscated for someone from the outside.} % LG: I added more details
For our survey we classify control paradigms for a creation process as (1) how is input given by an artist, (2) what type of content does an artist input, (3) where does the input have an effect spatially on the canvas and (4) when can input be given in the timeline of the creation process? Next, we categorize control mechanisms on an algorithmic level and on their input modes and primary effect into exemplars, parameterization, handling, filling, guiding, and placing interactions. We organise our analysis of the state of the art along these categories as well as based on the pattern design features they enable, such as repetition, frames, curves, directionality or single visual accents. The focus on design features allows for the integration and comparison of various techniques, such as procedural and data-driven approaches. For a better understanding of the potential of current techniques for creative creation and for making such an investigation more manageable, we motivate our discussion on the creative means of navigation, transparency, variation, and stimulation. 
% LG: Shall we better leave the following sentence out? Is it too grand?
All in all, we hope to inspire innovation for artist-centered creation processes on a grander scheme.




    % \note{This is currently a summary of the STAR for communication purposes - not the abstract:}

    % Supporting artists with meaningful digital tools for creative creation is an ongoing research challenge and spans over various disciplines. On an algorithmic level, most solutions focus on adding singular features and control mechanisms. Little attention is paid to novel methods that can complement each other as part of a cohesive pipeline and creative workflow. Towards the goal of enabling a creative workflow, we survey current methods in the scope of creative pattern generation and their control mechanisms. We classify techniques and discuss their potential to support creative creation with newly developed criteria. 
    % \newline
    
    % The specific goal of two-dimensional and creative pattern generation, offers a complex design challenge. On the one hand, the repetitive nature of a pattern is well represented with algorithmic creation, especially with procedural representations. On the other hand, the design space of possible pattern designs ranges from realistic, over abstracted to artistic. Here, considerations about a creative workflow and the need to go beyond automation rise and meaningful control needs to be given to artists. Typically however, with the power of procedural generation comes difficult controllability. Combining the algorithmic nature of procedural generation with novel forms of controllability has great capabilities and potential to offer truly novel benefits to traditional creation processes. 
    % \newline
    
    % For analyzing the state of the art and a better understanding of creative creation with digital tools, we dissect a creation process into to the methodologies of how, what, where and when. To relate control mechanisms on algorithmic level to the stages of a creation process, we classify specific control mechanisms into exemplars, parameterization, handling, filling, guiding and placing interactions. For making the domain of creative creation more manageable, the creative means of navigation, transparency, variation and stimulation are defined and linked to the control mechanisms.
    % \newline
    
    % With these criteria, we survey the most recent algorithmic representations for creative pattern generation. For this chosen design goal, the analysis bridges between various techniques such as procedural and data-driven approaches. We identify open issues and sketch out promising research directions towards the unification of different building blocks to a coherent pipeline. All in all, we hope to inspire innovation for artist-centered creation processes on a grander scheme.

%-------------------------------------------------------------------------
%  ACM CCS 1998
%  (see https://www.acm.org/publications/computing-classification-system/1998)
% \begin{classification} % according to https://www.acm.org/publications/computing-classification-system/1998
% \CCScat{Computer Graphics}{I.3.3}{Picture/Image Generation}{Line and curve generation}
% \end{classification}
%-------------------------------------------------------------------------
%  ACM CCS 2012
% (see https://www.acm.org/publications/class-2012)
%The tool at \url{http://dl.acm.org/ccs.cfm} can be used to generate
% CCS codes.
%Example:
% \begin{CCSXML}
% <ccs2012>
% <concept>
% <concept_id>10010147.10010371.10010352.10010381</concept_id>
% <concept_desc>Computing methodologies~Collision detection</concept_desc>
% <concept_significance>300</concept_significance>
% </concept>
% <concept>
% <concept_id>10010583.10010588.10010559</concept_id>
% <concept_desc>Hardware~Sensors and actuators</concept_desc>
% <concept_significance>300</concept_significance>
% </concept>
% <concept>
% <concept_id>10010583.10010584.10010587</concept_id>
% <concept_desc>Hardware~PCB design and layout</concept_desc>
% <concept_significance>100</concept_significance>
% </concept>
% </ccs2012>
% \end{CCSXML}

% \ccsdesc[300]{Computing methodologies~Collision detection}
% \ccsdesc[300]{Hardware~Sensors and actuators}
% \ccsdesc[100]{Hardware~PCB design and layout}

\begin{CCSXML}
<ccs2012>
<concept>
<concept_id>10010147.10010371</concept_id>
<concept_desc>Computing methodologies~Computer graphics</concept_desc>
<concept_significance>500</concept_significance>
</concept>
<concept>
<concept_id>10003120.10003123.10010860</concept_id>
<concept_desc>Human-centered computing~Interaction design process and methods</concept_desc>
<concept_significance>500</concept_significance>
</concept>
<concept>
<concept_id>10003120.10003123.10011758</concept_id>
<concept_desc>Human-centered computing~Interaction design theory, concepts and paradigms</concept_desc>
<concept_significance>500</concept_significance>
</concept>
<concept>
<concept_id>10003120.10003123.10011760</concept_id>
<concept_desc>Human-centered computing~Systems and tools for interaction design</concept_desc>
<concept_significance>300</concept_significance>
</concept>
</ccs2012>
\end{CCSXML}

\ccsdesc[500]{Computing methodologies~Computer graphics}
\ccsdesc[500]{Human-centered computing~Interaction design process and methods}
\ccsdesc[500]{Human-centered computing~Interaction design theory, concepts and paradigms}
\ccsdesc[300]{Human-centered computing~Systems and tools for interaction design}


\printccsdesc   

\end{abstract}

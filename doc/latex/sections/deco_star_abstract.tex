%!TEX root = deco_star.tex

\begin{abstract}

    \note{This is currently a summary of the STAR for communication purposes - not the abstract:}

    Supporting artists with meaningful digital tools for creative creation is an ongoing research challenge and spans over various disciplines. On an algorithmic level, most solutions focus on adding singular features and control mechanisms. Little attention is paid to novel methods that can complement each other as part of a cohesive pipeline and creative workflow. Towards the goal of enabling a creative workflow, we survey current methods in the scope of creative pattern generation and their control mechanisms. We classify techniques and discuss their potential to support creative creation with newly developed criteria. 
    \newline
    
    The specific goal of two-dimensional and creative pattern generation, offers a complex design challenge. On the one hand, the repetitive nature of a pattern is well represented with algorithmic creation, especially with procedural representations. On the other hand, the design space of possible pattern designs ranges from realistic, over abstracted to artistic. Here, considerations about a creative workflow and the need to go beyond automation rise and meaningful control needs to be given to artists. Typically however, with the power of procedural generation comes difficult controllability. Combining the algorithmic nature of procedural generation with novel forms of controllability has great capabilities and potential to offer truly novel benefits to traditional creation processes. 
    \newline
    
    For analyzing the state of the art and a better understanding of creative creation with digital tools, we dissect a creation process into to the methodologies of how, what, where and when. To relate control mechanisms on algorithmic level to the stages of a creation process, we classify specific control mechanisms into exemplars, parameterization, handling, filling, guiding and placing interactions. For making the domain of creative creation more manageable, the creative means of navigation, transparency, variation and stimulation are defined and linked to the control mechanisms.
    \newline
    
    With these criteria, we survey the most recent algorithmic representations for creative pattern generation. For this chosen design goal, the analysis bridges between various techniques such as procedural and data-driven approaches. We identify open issues and sketch out promising research directions towards the unification of different building blocks to a coherent pipeline. All in all, we hope to inspire innovation for artist-centered creation processes on a grander scheme.

%-------------------------------------------------------------------------
%  ACM CCS 1998
%  (see https://www.acm.org/publications/computing-classification-system/1998)
% \begin{classification} % according to https://www.acm.org/publications/computing-classification-system/1998
% \CCScat{Computer Graphics}{I.3.3}{Picture/Image Generation}{Line and curve generation}
% \end{classification}
%-------------------------------------------------------------------------
%  ACM CCS 2012
(see https://www.acm.org/publications/class-2012)
%The tool at \url{http://dl.acm.org/ccs.cfm} can be used to generate
% CCS codes.
%Example:
\begin{CCSXML}
<ccs2012>
<concept>
<concept_id>10010147.10010371.10010352.10010381</concept_id>
<concept_desc>Computing methodologies~Collision detection</concept_desc>
<concept_significance>300</concept_significance>
</concept>
<concept>
<concept_id>10010583.10010588.10010559</concept_id>
<concept_desc>Hardware~Sensors and actuators</concept_desc>
<concept_significance>300</concept_significance>
</concept>
<concept>
<concept_id>10010583.10010584.10010587</concept_id>
<concept_desc>Hardware~PCB design and layout</concept_desc>
<concept_significance>100</concept_significance>
</concept>
</ccs2012>
\end{CCSXML}

\ccsdesc[300]{Computing methodologies~Collision detection}
\ccsdesc[300]{Hardware~Sensors and actuators}
\ccsdesc[100]{Hardware~PCB design and layout}


\printccsdesc   

\end{abstract}

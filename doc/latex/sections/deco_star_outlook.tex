
\section{Outlook}
\label{sec:outlook}

The review of the state of the art shows that there are various limitations and possibilities for future work within the specific contexts of the work. However, there are also novel paradigms for creative control and their underlying algorithms that uniquely add to the state of the art. The most prominent development is the integration of machine learning techniques and so-called mixed-initiative algorithms and possibilities of the usage of semantic attributes.

In addition collaboration is a valuable future line of investigation for enabling creative work. With regard to technology, more and more aspects of common tools either are fully browser-based or are in some way connected to the cloud-based storage of assets, settings and results; therefore, they function online and are easily shared. Collaboration is closely connected to the previously discussed issue of navigation histories. This is not only relevant for individual work processes but also for more general production pipelines in a commercial context. In this regard, the sharing and collaborative work on iterations, which involves multiple persons referencing different versions, is essential. Some work has been done (e.g.,~\cite{talton_2009_emw, salvati_2015_mcm,oleary_2018_csi}) but further investigations of collaboration for creative control are called for. 

As discussed in \Cref{sec:models}, control techniques are closely inter-related with the representation of the underlying models. Therefore, a more unified development of models across research communities would be beneficial. Expert knowledge of usability should especially be considered. However, further automation for the creation of complex decorative models also poses interesting challenges, such as abstraction \cite{nan_2011_cgr}, symmetry computation~\cite{cullen_2011_sh} and design space variations.


\subsection{Mixed Initiative Interfaces}
\label{subsec:analysis_outlook_mixed_initiative_interfaces}

% This is actually quite important and could/should be further elaborated.
The procedural content generation (PCG) community for games is pushing the general integration of artificial intelligence (AI) into a procedural creation process. A new paradigm of \textit{mixed-initiative creative interfaces} is rising and is actively fostered, as an ACM Conference on Human Factors in Computing Systems (CHI) workshop under the same name in 2017 shows~\cite{deterding_2017_mci}. As the workshop summary states, it is the goal to ``put human and computer in a tight interactive loop where each suggests, produces, evaluates, modifies, and selects creative outputs in response to the other.'' In order to achieve this, AI enables computer agency, and novel interfaces enable collaboration between computers and human users. In regard such AI systems and the integration of agency and automation \citeauthor*{heer_2019_apa}~\cite{heer_2019_apa} discusses three case studies and poses several open questions for further developments.

% The workshop brought PCG and interaction design researchers together, stressing the importance of bridging disciplines. Similarly to games, the context of creative pattern generation also constitutes a challenging but fruitful testing ground for the investigation of mixed-initiative creative interfaces and for the task of balancing artist control and automation, as this survey shows. In general, the involvement of the computer graphics community with its various topics and rich algorithmic knowledge would be promising.

\subsection{Semantic Attributes}
\label{subsec:analysis_outlook_semantic-attributes}

The usage of semantic attributes presents a highly intuitive navigation technique, which so far has been successfully applied in the context of shape modifications, for example by \citeauthor*{yumer_2015_sse}~\cite{yumer_2015_sse}.

In the context of complex patterns, procedural textures constitute the most related field of investigation. For the control of procedural textures, methods are based on the analysis and description of texture in regard to human perception, which has a long research tradition. In his influential work, \citeauthor*{julesz_1981_tte}~\cite{julesz_1981_tte} defines textons as the basic units of pre-attentive human texture perception. Since then, this line of research has continued, and texture descriptions with perceptual \cite{liu_2015_vpp} and semantic~\cite{matthews_2013_eta,cimpoi_2014_dtw} attributes have been investigated. \citeauthor*{dong_2017_ptg}~\cite{dong_2017_ptg} and \citeauthor*{liu_2018_ppt}~\cite{liu_2018_ppt} employed such features in first experiments for the navigation of a procedural texture space and for the generation of suitable textures by given features. % As texture generation method the authors apply a Generative Adversarial Networks, which includes semantic attributes in the texture generation process. 
However, the results of such studies are still limited and of varying quality - and the authors themselves \cite{liu_2018_ppt} call their results experimental.

Nonetheless, these works present an interesting approach that is worth further investigation. Because many pattern designs are structured and follow an internal logic, it seems feasible to come up with a collection of suitable attributes. For example an ornamental design space is much smaller in comparison to all ``textures in the wild''~\cite{cimpoi_2014_dtw}, and ornamentation could constitute a valuable context for further investigations into the incorporation of semantic attributes into a creative creation process.

% % \minortodo[inline]{Add Semantic aware methods for evolutionary art} ??
% https://dl.acm.org/doi/10.1145/2576768.2598293

% \insertref[inline]{\cite{joppi_2019_tri}}